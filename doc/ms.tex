% mnras_template.tex
%
% LaTeX template for creating an MNRAS paper
%
% v3.0 released 14 May 2015
% (version numbers match those of mnras.cls)
%
% Copyright (C) Royal Astronomical Society 2015
% Authors:
% Keith T. Smith (Royal Astronomical Society)

% Change log
%
% v3.0 May 2015
%    Renamed to match the new package name
%    Version number matches mnras.cls
%    A few minor tweaks to wording
% v1.0 September 2013
%    Beta testing only - never publicly released
%    First version: a simple (ish) template for creating an MNRAS paper

%%%%%%%%%%%%%%%%%%%%%%%%%%%%%%%%%%%%%%%%%%%%%%%%%%
% Basic setup. Most papers should leave these options alone.
\documentclass[a4paper,fleqn,usenatbib]{mnras}

% MNRAS is set in Times font. If you don't have this installed (most LaTeX
% installations will be fine) or prefer the old Computer Modern fonts, comment
% out the following line
\usepackage{newtxtext,newtxmath}
% Depending on your LaTeX fonts installation, you might get better results with one of these:
%\usepackage{mathptmx}
%\usepackage{txfonts}

% Use vector fonts, so it zooms properly in on-screen viewing software
% Don't change these lines unless you know what you are doing
\usepackage[T1]{fontenc}
\usepackage{ae,aecompl}


%%%%% AUTHORS - PLACE YOUR OWN PACKAGES HERE %%%%%

% Only include extra packages if you really need them. Common packages are:
\usepackage{graphicx}	% Including figure files
\usepackage{amsmath}	% Advanced maths commands
\usepackage{amssymb}	% Extra maths symbols

\usepackage{microtype}

%%%%%%%%%%%%%%%%%%%%%%%%%%%%%%%%%%%%%%%%%%%%%%%%%%

%%%%% AUTHORS - PLACE YOUR OWN COMMANDS HERE %%%%%

% Please keep new commands to a minimum, and use \newcommand not \def to avoid
% overwriting existing commands. Example:
%\newcommand{\pcm}{\,cm$^{-2}$}	% per cm-squared

%%%%%%%%%%%%%%%%%%%%%%%%%%%%%%%%%%%%%%%%%%%%%%%%%%

%%%%%%%%%%%%%%%%%%% TITLE PAGE %%%%%%%%%%%%%%%%%%%

% Title of the paper, and the short title which is used in the headers.
% Keep the title short and informative.
\title[]
{Inferring obscuring material from transit light curves}
    
\author[Brewer and Knuth]{%
  Brendon~J.~Brewer$^{1}$\thanks{To whom correspondence should be addressed. Email: {\tt bj.brewer@auckland.ac.nz}},
  Kevin H. Knuth$^2$
  \medskip\\
  $^1$Department of Statistics, The University of Auckland, Private Bag 92019,
        Auckland 1142, New Zealand\\
  $^2$Physics Department, University at Albany (SUNY), Albany, NY 12205, USA}
% These dates will be filled out by the publisher
\date{}

% Enter the current year, for the copyright statements etc.
\pubyear{2016}

% Don't change these lines
\begin{document}
\label{firstpage}
\pagerange{\pageref{firstpage}--\pageref{lastpage}}
\maketitle

% Abstract of the paper
\begin{abstract}
\end{abstract}

% Select between one and six entries from the list of approved keywords.
% Don't make up new ones.
\begin{keywords}
methods: data analysis --- methods: statistical
\end{keywords}

%%%%%%%%%%%%%%%%%%%%%%%%%%%%%%%%%%%%%%%%%%%%%%%%%%

%%%%%%%%%%%%%%%%% BODY OF PAPER %%%%%%%%%%%%%%%%%%

\section{Introduction}


\section{Hypothesis Space}
Our modelling assumptions are intended as a compromize between flexibility
(so we can potentially infer complex structures) and computational
tractability.

Consider a coordinate system $(x,y)$ in units of the star's radius.
The surface brightness profile
$S(x, y)$ of the star uses the limb darkening profile
\begin{align}
S(x, y) &\propto
    \left\{
        \begin{array}{lr}
            1 - b\left(1 - \sqrt{1 - (x^2 + y^2)}\right),   & x^2+y^2 \leq 1\\
            0, & \textnormal{otherwise}
        \end{array}
    \right.
\end{align}
where $b$ is a limb darkening parameter.

The density profile of the obscuring material is composed of a sum of
$N$ circular ``blobs'' based on the blob profile
used by \citet{lensing2}. A single blob of width $w$ and total ``mass'' $M$
(the integral of the density profile),
positioned at the origin, would have density profile
\begin{align}
\rho(x, y) &= \left\{
        \begin{array}{lr}
            \frac{2M}{\pi w^2}\left(1 - \frac{r^2}{w^2}\right), & r \leq w\\
            0, & \textnormal{otherwise}
        \end{array}\right.
\end{align}
where $r = \sqrt{x^2 + y^2}$. The total density profile of $N$ blobs,
with masses $\{M_i\}$ and centers $\{(x_i, y_i)\}$, is
\begin{align}
\rho(x, y) &= \sum_{i=1}^N
        \left\{
        \begin{array}{lr}
            \frac{2M_i}{\pi w_i^2}
                \left(1 - \frac{r_i^2}{w_i^2}\right), & r_i \leq w_i\\
            0, & \textnormal{otherwise}
        \end{array}\right.
\end{align}
where $r_i = \sqrt{(x - x_i)^2 + (y - y_i)^2}$ is the distance of a point
in the $x-y$ plane from the central position $(x_i, y_i)$ of blob $i$.

If the blob density is taken as the optical depth, then the overall image
of the star is
\begin{align}
I(x, y) &= S(x, y)e^{-\rho(x,y)}
\end{align}
and the observed total flux of the star is
\begin{align}
F &= \int_{-\infty}^{\infty}\int_{-\infty}^{\infty}
        I(x, y) \, dx \, dy.
\end{align}
We define the normalizing constant of the star's surface brightness profile
such that the total flux of the star would be 1 in the absence of any
obscuration.

\section*{Acknowledgements}
This work was funded by a Marsden Fast Start grant from the Royal Society of
New Zealand.

%%%%%%%%%%%%%%%%%%%%%%%%%%%%%%%%%%%%%%%%%%%%%%%%%%

%%%%%%%%%%%%%%%%%%%% REFERENCES %%%%%%%%%%%%%%%%%%

% The best way to enter references is to use BibTeX:

\bibliographystyle{mnras}
\bibliography{references} % if your bibtex file is called example.bib


%%%%%%%%%%%%%%%%%%%%%%%%%%%%%%%%%%%%%%%%%%%%%%%%%%

%%%%%%%%%%%%%%%%% APPENDICES %%%%%%%%%%%%%%%%%%%%%

%\appendix
%\section{Some extra material}

%%%%%%%%%%%%%%%%%%%%%%%%%%%%%%%%%%%%%%%%%%%%%%%%%%


% Don't change these lines
\bsp	% typesetting comment
\label{lastpage}
\end{document}

% End of mnras_template.tex
