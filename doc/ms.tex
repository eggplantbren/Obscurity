\documentclass[useAMS,usenatbib]{mn2e}

\voffset=-0.8in

% Packages:
\usepackage{graphicx}
\usepackage{amsmath}
\usepackage{xspace}
%\usepackage{dsfont}
\usepackage[utf8]{inputenc}
\usepackage{float}
\usepackage{color}

\usepackage{epstopdf}

% For revisions
\newcommand{\revisions}{}
\newcommand{\revisionstwo}{}

\title[]
{Inferring obscuring material from transit light curves}
    
\author[Brewer and Knuth]{%
  Brendon~J.~Brewer$^{1}$\thanks{To whom correspondence should be addressed. Email: {\tt bj.brewer@auckland.ac.nz}},
  Kevin H. Knuth$^2$
  \medskip\\
  $^1$Department of Statistics, The University of Auckland, Private Bag 92019, Auckland 1142, New Zealand\\
  $^2$Department of Physics, SUNY Albany}

%%%%%%%%%%%%%%%%%%%%%%%%%%%%%%%%%%%%%%%%%%%%%%%%%%%%%%%%%%%%%%%%%%%%%%%%%%%%%%

\begin{document}
             
\date{}
             
\maketitle

\label{firstpage}

\begin{abstract}
Abstract
\end{abstract}


\begin{keywords}
methods: data analysis --- methods: statistical
\end{keywords}

\section{Introduction}


\section{Hypothesis Space}
Our modelling assumptions are intended as a compromize between flexibility
(so we can potentially infer complex structures) and computational
tractability.

Consider a coordinate system $(x,y)$ in units of the star's radius.
The surface brightness profile
$S(x, y)$ of the star uses the limb darkening profile
\begin{align}
S(x, y; b) &=
    \left\{
        \begin{array}{lr}
            1 - b\left(1 - \sqrt{1 - (x^2 + y^2)}\right),   & x^2+y^2 \leq 1\\
            0, & \textnormal{otherwise}
        \end{array}
    \right.
\end{align}
where $b$ is a limb darkening parameter.

The density profile of the obscuring material is composed of a sum of
$N$ circular ``blobs'' based on the surface brightness profile


\section*{Acknowledgements}
This work was funded by a Marsden Fast Start grant from the Royal Society of
New Zealand.


\begin{thebibliography}{99}
\end{thebibliography}


\end{document}

